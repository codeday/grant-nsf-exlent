%
% No references allowed
%
\section{Data Management Plan}

\subsection{Types of Data}
\begin{itemize}
    \item \textbf{Planning Materials:} This will include calendars for activities to be undertaken in the proposed project, literature reviews, information about curricula from partner colleges, etc. which are generated by both the PIs and the evaluator.

    \item \textbf{Program Materials:} Mentor training courses, student training and on-boarding courses, meeting agenda outlines, and cohort workshops will be created by the PIs.
    
    \item \textbf{Open-Source Contributions:} Students will create code and other work products during each micro-internship in which they participate. Students will share their contributions back to the original open-source project licensed with the original project's OSI-approved license.

    \item \textbf{Participant Data:} Qualitative and quantitative data will be collected by the PIs and the evaluator through surveys, systematic observation, and interviews with faculty, students, mentors, and other industry representatives. The evaluator and PIs will collaborate to develop appropriate measures of student skills and employability and to determine survey and interview protocols. Standard quantitative analysis methods will be used for surveys and systematic observation data while qualitative methods will be used for interviews.
    
    \item \textbf{Analysis, Reports, and Publications:} One method of distribution this project will use is by sharing analyses, reports, and publications. Additionally, such reports may be shared internally to improve the program.
\end{itemize}


\subsection{Data Formats}
\begin{itemize}
    \item \textbf{Planning Materials, Program Materials, Analysis, Reports, and Publications:} Text documents will be shared as PDFs, Word (.docx) files, and/or \LaTeX code. Multimedia will be made available as .mp4 video files and video edit sources will be stored as OpenTimelineIO (.otio) files.
    
    \item \textbf{Participant Data:} Datasets will either be stored directly as spreadsheets/CSVs or JSON files, or stored in a live PostgreSQL/MySQL database and archived in one of the above formats. Qualitative observations will be archived as PDFs, Word (.docx) files, and/or \LaTeX code. The evaluation code will be in Python notebooks or source code (.py, .ts, etc.).
    
    \item \textbf{Open Source Contributions:} Students will submit open source contributions as ``Pull Requests'' or ``Merge Requests'' directly on an open-source hosting platform; they can be downloaded as complete source code archives or as Git .patch files.
\end{itemize}


\subsection{Data Access and Confidentiality}
\begin{itemize}
    \item \textbf{Planning Materials, Program Materials, Analysis, and Reports:} Final copies of all such materials generated as part of the proposed program will be made available for free use, modification, and redistribution under an OSI-approved open source license (for code) or a Creative Commons Attribution/Attribution-ShareAlike/Attribution-NonCommercial-ShareAlike license (for non-code). Program materials that were not created as part of this program and which constitute MinT's existing IP may not be shared. All of CodeDay's existing IP is already available under an OSI or Creative Commons license.

    \item \textbf{Publications:} The PIs and/or the evaluator will publish all final copies open access. Preprints may be shared on ArXiV or a similar preprint archive.
    
    \item \textbf{Participant Data:} Live electronic data will be stored in PostgreSQL or MySQL databases, which are backed up regularly and secured in compliance with industry-standard SOC 2 security requirements. Databases are only accessible from an internal network with a password which is only available to production software running in the same security environment; neither PIs nor the evaluator will not have direct access to data. Code that accesses production data will be subject to code reviews prior to deployment to production. Hard copy data containing PII will be stored by either the PI or evaluators in locked filing cabinets in locked offices. Anonymized datasets will be made freely available to the research community for analysis and redistribution, subject to IRB approval.

    \item \textbf{Open Source Contributions:} Students will make their completed contributions available publicly under an OSI-approved open source license and will be viewable from an open-source hosting platform (such as GitHub, Gitea, Gitlab, Bitbucket, etc.) selected by the students or the associated project.
\end{itemize}

\subsection{Long-Term Preservation}
\begin{itemize}
    \item \textbf{Planning Materials, Program Materials, Analysis, and Reports:} Copies will be archived for at least five years following the completion of the project on CodeDay's secure archive infrastructure. Copies will be sent to Washington State Center of Excellence in Information \& Computing Technology and the AppConnect NW network for possible inclusion in these agencies’ resource libraries.

    \item \textbf{Publications:} One long-term strategy for archiving the outcomes of this project will be through publication in journals and conference proceedings. Publications will be available indefinitely in the journal in which they are published, and preprints may be archived indefinitely in ArXiV or a similar preprint archive.

    \item \textbf{Participant Data:} All data will be electronically archived on CodeDay's secure archive infrastructure for at least five years following the completion of the project. Publicly released data will be archived indefinitely on the Center for Open Science's OSF Storage, and data relevant to publications will also be included in the journal's data repository, when available.

    \item \textbf{Open Source Contributions:} Student contributions will be archived by the open-source hosting platform and to other mirroring platforms.
\end{itemize}