\begin{center}
\uppercase{\textbf{Project Summary}}
\end{center}

\textbf{Overview} \\
This ExLENT \textbf{Beginnings Track} proposal will create an experiential pathway for Computer Science (CS) students interested in the \textbf{emerging technology field of Artificial Intelligence and Machine Learning (AI/ML)} to work toward a career through micro-internships and structured mentoring. The program elements will ultimately help \textbf{150 diverse and under-represented students from Community and Technical Colleges (CTCs)} in 3 annual cohorts navigate and launch careers in AI/ML through:

\begin{enumerate}
    \item \textbf{Engagement with AI/ML employers} to recruit mentors and align the program to workforce needs.
    \item \textbf{Creation of AI/ML Open Source micro-internship.} Students will complete 1-2 ``micro-internships'' working on real-world contributions to open source AI software under the guidance of industry mentors.
    \item \textbf{Structured AI/ML career mentorship.} Students will be matched with two industry mentors, meet virtually monthly with each, and submit structured reflections.
    \item \textbf{Cohort activities.} Students will participate in training, attend monthly workshops, and be coached to apply to 3 job opportunities each month.
\end{enumerate}

\textbf{Intellectual Merit} \\
Coursework enables students to learn the skills needed for a career, but industry engagement through internships has been shown to help students understand how to apply what they have learned, ultimately improving graduation and job placement rates. Internships are limited by industry engagement and most students never obtain one, especially the diverse, low-income students who attend CTCs. This proposal will provide CTC students with two one-month micro-internships working on open-source AI/ML software supervised by industry mentors. In addition to mentor engagement with projects, students also receive wrap-around career mentoring throughout the school year. Mentoring is common for new entrants to many fields, and studies have shown improvements to retention, academic success, and employment after graduation, but mentoring is still uncommon in emerging technology fields, and even more so for underserved CTC students. This proposal will contribute by implementing a fully-fledged program informed by this research.

\textbf{Broader Impacts} \\
Although most students pursue a degree to improve their job prospects, many graduates report feeling unprepared for a career, especially those in emerging tech fields. Likewise, AI/ML employers have reported that new graduates lack the skills to succeed. While internships and mentorship have been shown to close this gap, these programs are expensive to scale due to the large upfront commitments from both industry and colleges, so few students can access them. CTCs are especially disadvantaged due to limited industry relationships and under funding. This proposal provides 150 diverse students at CTCs with internships and wrap-around mentoring, and creates an AI/ML career pathway previously not available to CTC students. A sustainable, scalable program and published details for replication will provide these experiences to thousands of future CTC students in the long-term.


