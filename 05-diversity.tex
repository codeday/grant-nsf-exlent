\section{Building an Inclusive and Diverse STEM Workforce}

The US and local tech industries have had a long history of only recruiting from elite institutions, and many CTC students are put in a place of outsiders looking in. This is especially true when it comes to highly paid, in-demand technical positions. For example, Black and African Americans account for 5.7\% of technical roles at Microsoft, Hispanic and Latinx accounts for 6.8\%, and 25.8\% for women, \cite{microsoftDiversityInclusionReport} all three severely short of the US population as a percentage and more so in the context of CTC students. This project specifically and exclusively works with highly diverse CTCs. Nationwide, students who attend CTCs are: 56\% students of color, 55\% female, 25\% first-generation immigrants, 33\% receiving Pell grants, 15\% parents, and 10\% veterans.\cite{AlmanacAmericanEducation,phillippeNationalProfileCommunity2005} 

\subsubsection{Creating On-Ramps for Diverse Individuals Into Careers in Emerging Technology Fields}

Internships are the single largest factor in whether students obtain a job post-graduation \cite{callananAssessingRoleInternships2004,jonesTransformingCurriculumPreparing2002,knouseRelationCollegeInternships1999,saltikoffPositiveImplicationsInternships2017,stepanovaHiringCSGraduates2021}, their starting salary and time spent looking for a job.\cite{gaultUndergraduateBusinessInternships2000} Despite their importance, access to internships is unequal. Studies find that half as many low-income students find internships,\cite{kapoorExploringParticipationCS2020,singerLowerIncomeStudentsBig2023} that race and gender likely also play a role, and that students attending community colleges have additional difficulties because their schools lack brand name recognition.\cite{menezesOpenSourceInternshipsIndustry,lohrEngineUpwardMobility2022} In one large CTC system, less than 10\% of college CS majors were able to secure an internship.\cite{lohrEngineUpwardMobility2022} By using paid micro-internships as the method of experiential learning, and offering them to diverse students who are least likely to be offered an internship, we can give these students the same powerful boost to learning\cite{ellisPowerOpenSource2021,hislopStudentReflectionsLearning2020,hislopOpenSourceExtracurricular2019,postnerSurveyInstructorsExperiences2018} and careers\cite{saltikoffPositiveImplicationsInternships2017}. This responds to the ExLENT program’s criterion that the project must create on-ramps for diverse individuals into careers in emerging technology fields.

\subsubsection{Reducing Barriers for Groups Historically Underrepresented in STEM}

A majority of historically underrepresented students are unable to participate in internships because of competing life priorities, specifically: because they are a primary caretaker for a family member, and/or because they need to work to earn money for their family. \cite{kapoorBarriersSecuringIndustry2020} This proposal removes both barriers by requiring a limited time commitment –- 5 hours per month for mentor meetings and cohort activities + 8-10 hours per week during the one-month micro-internship -– and by providing an industry-competitive wage for all time spent in the program, including mentor meetings and cohort activities. This responds to the ExLENT program’s criterion that the project must reduce barriers so that members of groups historically underrepresented or under-served in STEM can acquire the training and learning needed for careers in emerging technology.

\subsubsection{Open Source Work Increases Diversity}

The proposed model solves another problem for driving more students from underrepresented groups to pursue AI careers: many of these students are disinterested in technical careers because they view technical work as not contributing to society. Studies suggest that asking students to make contributions to open source software can avoid this problem, improving retention and career outcomes.\cite{ellisPowerOpenSource2021,morelliRevitalizingComputingEducation2009,ellisCanHumanitarianOpensource2007}

\subsubsection{Provide Participants With a Path Beyond the ExLENT Program}

Another reason students have difficulty finding employment is not adequately preparing to find a job because they do not know what to do. One study found that only 45\% of students prepare applications, 44\% work on projects outside of schoolwork, and 27\% practice interviewing. 23\% of students reported doing absolutely nothing, believing doing well in school was all that was important. \cite{kapoorExploringParticipationCS2020} The mentoring component of the pathway created by this program pairs each student with two mentors from the technology industry, and the curriculum addresses these common career misunderstandings, in order to improve the chances that each student is hired. Students with mentors are also twice as likely as their peers to complete their degree.\cite{salinitriEffectsFormalMentoring2005,campbellFacultyStudentMentor1997b,mangoldWhoGoesWho2002} This responds to the ExLENT program’s criterion that the project must provide participants with a path beyond the ExLENT program.