\section{Project Overview, Rationale and Importance}

The "CTCtoAI" ExLENT \textbf{Beginnings Track} project from Student Research and Development (dba CodeDay) and Mentors in Tech (MinT) will build a diverse Artificial Intelligence and Machine Learning (AI/ML) workforce by providing 150 diverse students from Community and Technical Colleges (CTCs) with practical skills through micro-internships and mentoring, ensuring they are well-prepared for AI/ML careers. This initiative addresses the need for a more inclusive and skilled workforce in these rapidly evolving fields, enhancing our nation's competitiveness and innovation in the emerging technology of AI/ML.

The proposed program will accept 3 annual cohorts of 50 students each (150 total) into a guided AI/ML pathway. Each cohort will complete 1-2 Open Source Micro-Internships, meet monthly with two assigned career mentors for 8 months, and participate in cohort workshops and meetings. Students will be paid a stipend of \$3,300 (approximately \$25 per hour for the total estimated time commitment of 130 hours). Alignment with the ExLENT program’s solicitation-specific criteria is detailed in "Experiental Learning Activities" and "Building an Inclusive and Diverse STEM Workforce". 

\begin{figure}[H]
    \begin{ganttchart}[
        vgrid,
        x unit=1cm,
        title/.style={draw=none, fill=none},
        bar/.append style={rounded corners=3pt},
        bar left shift=.15,
        bar right shift=-.15,
        bar top shift=.3,
        bar height=.3,
        group top shift=.9,
        group height=.1,
        group peaks height=.2,
        group left shift=0,
        group right shift=0,
        group peaks tip position=0,
        milestone top shift=0.24
    ]{1}{12}
        \gantttitlelist[
            title list options=%
                {var=\y, evaluate=\y as \x%
                using "\pgfcalendarmonthshortname{\y}"}
            ]{9,10,11,12,1,2,3,4,5,6,7,8}{1}

        \\

        \ganttbar{Cohort Activities}{1}{10}
        
        \\
        
        \ganttgroup[inline]{\normalfont (choose one)}{2}{3}
        \ganttgroup[inline]{\normalfont (choose one)}{6}{8}
        
        \\
        
        \foreach \n [count=\ni] in {2,3,6,7,8}{
            \ifnum\ni=1
                \ganttbar{Micro-Internship}{\n}{\n.99}
            \else
                \ganttbar{}{\n}{\n.99}
            \fi
            \ganttmilestone[milestone/.append style={shape=ellipse, anchor=east, xshift=-18pt, xscale=0.15, yscale=0.3}]{}{\n}
            \ganttmilestone[milestone/.append style={shape=ellipse, anchor=east, xshift=-12pt, xscale=0.15, yscale=0.3}]{}{\n}
            \ganttmilestone[milestone/.append style={shape=ellipse, anchor=east, xshift=-7pt, xscale=0.15, yscale=0.3}]{}{\n}
        }

        \\

        \ganttbar{Career Mentoring}{3}{10}
        \foreach \n in {3,...,10}{
            \ganttmilestone[milestone/.append style={shape=ellipse, anchor=east, xshift=-15pt, xscale=0.15, yscale=0.3}]{}{\n}
            \ganttmilestone[milestone/.append style={shape=ellipse, anchor=east, xshift=-8pt, xscale=0.15, yscale=0.3}]{}{\n}
        }
    \end{ganttchart} \\
    \centering \small \textit{
        $\bullet$ = meeting with industry mentor
    }
    
    \caption{Calendar for each of the three yearly cohorts.}
    \label{fig:cohortCalendar}
\end{figure}

The objectives of this project are:

\begin{itemize}
    \item Create an experiential learning opportunity in AI/ML using micro-internships.
    \item Instill a career-oriented outlook in students through cohort activities and a central focus on industry mentorship.
    \item Develop a talent pipeline in partnership with academia and industry which comprises coursework, experiential learning, mentorship, and career opportunities.
    \item Promote a diverse STEM workforce through programs tailored to the needs of the diverse and under-served students attending America’s CTCs.
\end{itemize}

\subsection{Motivation}

Artificial Intelligence and Machine Learning are rapidly evolving technologies that have already found broad applications in most industries. Developing a skilled workforce in these fields is crucial because they have the potential to address many of the most complex challenges our nation faces, from climate change to digital security. At the same time, many in academia and the industry have raised concerns about bias, as these technologies can inadvertently perpetuate or even exacerbate existing societal prejudices; to address this issue effectively, it is crucial to incorporate diverse voices and perspectives into this emerging technology.

Washington State is well-positioned to become a leader in the AI/ML sector, but the labor supply does not meet the demand. Washington State is home to companies like Microsoft and Amazon, some of the largest investors in AI/ML technologies, as well as leading research hubs and incubators like AI2. However, technology industry leaders report significant difficulty finding a local workforce to fill jobs in this emerging technology. \cite{soper_is_2023}

CTCs have the highest percentage of students from diverse and underrepresented  backgrounds: Black, Indigenous and People Of Color (BIPOC), women, low-income, and non-traditional (adults, veterans, working).\cite{AlmanacAmericanEducation,phillippeNationalProfileCommunity2005} Unfortunately, these nontraditional students are the furthest from opportunities in AI/ML, and tech in general,\cite{singerLowerIncomeStudentsBig2023,saldanhaBlackWorkersWomen2023} compared to their peers attending traditional 4-year college programs.\cite{lohrEngineUpwardMobility2022} Most CTC students have no experience with the tech hiring process and have no preparation on how the technology sector does hiring --- the unspoken rules, expectations, norms, and how to navigate the landscape successfully to land a high-paying job in emerging tech that enables their and their families’ economic and social mobility.

\subsection{Expected Outcomes}

This project will combine experiential learning through micro-internships with virtual mentorship to build a diverse workforce in the emerging technology field of Artificial Intelligence (AI). 

The project will deliver the following measurable outcomes over three years:

\begin{enumerate}
    \item Recruit 150 students who do not have access to traditional internships, and/or connections to the emerging tech industry of AI.
    \item Recruit and train industry mentors for micro-internships and career mentorship.
    \item 90\% of students complete 2 micro-internships (contributing to AI open source software).
    \item 90\% of students attend at least 12 one-on-one meetings with assigned industry mentors.
    \item Establish a peer cohort among students.
    \item Conduct a series of professional development workshops for cohorts.
    \item Collaborate with CTC faculty to identify opportunities to award academic credit to students for experiential learning.
    \item Identify and disseminate promising practices through conference presentations and publications.
\end{enumerate}

\subsection{Project Leadership Team}

PI \textbf{Tyler Menezes} is the founder and Executive Director of CodeDay, a worldwide nonprofit which has been featured in NPR, TechCrunch, Geekwire, Readwrite, and Forbes Magazine. The program has served more than 60,000 students and launched more than 40,000 tech and tech-adjacent careers since its founding in 2009. Before CodeDay, Tyler was the founder and VP of Engineering of a Y Combinator and venture-backed startup, and previously worked in AI/ML at several companies. His work on CodeDay has led to his recognition on Forbes Magazine’s “30 Under 30” and as Tech\&Learning Magazine “Most Innovative in EdTech”.

Co-PI \textbf{Kevin Wang} is a lifelong educator and engineer. He previously founded the Microsoft Philanthropies TEALS program that, over the past 12 years, has partnered with over 1,000 high schools across the country, a majority of which are low-income, to build and grow successful, sustainable, and diverse CS programs. TEALS taught over 100,000 high school students CS with thousands of industry volunteers investing millions of hours in the classroom. Kevin is able to bring his experience and best practices in building TEALS and working with an incredibly diverse set of schools and students as well as working with a multitude of industry partners and mentors into the MinT program. His work on TEALS has been featured in The New York Times, CNN, USA Today, Geekwire, Harvard Graduate School of Education Magazine, Univision, and was the subject of a Yale School of Management case study.

Information about the project's independent evaluator can be found under ``Evaluation''.