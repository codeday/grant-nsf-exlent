\section{Partnerships}

\begin{figure}[H]
    \fontsize{11pt}{11pt}\selectfont
    \begin{tabularx}{\textwidth}{ X c c c c }
        \toprule
        \textbf{Partner}
            & \textbf{Advisory}
            & \textbf{CTCs}
            & \textbf{Industry}
            & \textbf{Dissemination} \\
        
        & & (Students) & (Mentors) & \\
        \midrule \endhead
        \bottomrule\endfoot
    
        CCSF/UCSF BIL & $\bullet$ & & &  \\ \addlinespace \hline \addlinespace
        AppConnect NW & $\bullet$ & $\bullet$ & & \\ \addlinespace \hline \addlinespace
        SBCTC & $\bullet$ & $\bullet$ & & $\bullet$ \\ \addlinespace \hline \addlinespace
        WTIA & $\bullet$ & & $\bullet$  & \\ \addlinespace \hline\addlinespace
        WTB & & & & $\bullet$ \\ \addlinespace \hline \addlinespace
        CCBA & & & & $\bullet$ \\ \addlinespace \hline
        
    \end{tabularx}
    
    \caption{Overview of CTCtoAI partnerships}
    \label{fig:partnerRoles}
\end{figure}

\subsection{Partnership Vision and Goals}

A Washington State-wide collaboration between CTCs, the AI industry, and workforce development representatives will work together to provide students with accessible experiential learning opportunities and career mentorship which are tied cohesively to college curriculum and workforce needs. Dissemination partners, communities of practice, and collaborations with other NSF-funded projects will help ensure that the work is grounded in best practices and that results are shared broadly to make a larger impact.

The partners share a vision for an end-to-end pathway that takes students from school, to experiential learning, to mentorship, to the AI/ML workforce.

\subsection{Partner Roles, Responsibilities, and Benefits}

\subsubsection{City College of San Francisco (CCSF) \& University of California, San Francisco (UCSF), Building Inclusive Labs (BIL) Collaborative Project (DUE\#2055735/2055309)}

Building Inclusive Labs (BIL) is a collaborative initiative between a community college biotechnology program (CCSF) and a top-tier research university and technician employer (UCSF) whose goal is to build more inclusive workplace environments for community college students pursuing biotechnology careers. \textbf{Responsibilities:} The BIL project has committed to: providing their Workplace Navigation Training to students participating in CTCtoAI; providing their Mentor-Manager Training to industry mentors; and serving in an advisory role by fostering an experiential learning community of practice. \textbf{Benefits to Partner:} BIL is partnering with the CTCtoAI project as a way to disseminate their learnings and collaborate to expand their learnings in biotech into the new field of AI.


\subsubsection{AppConnect NW (DUE\#1700629)}

AppConnect NW is a consortium of nine Community and Technical Colleges in Washington State, created to bridge the gap between community colleges and the technology industry. \textbf{Responsibilities:} AppConnect NW will connect the project with educators and students at its member CTCs. The organization will also serve as the day-to-day implementation collaborator. AppConnect NW Faculty Associate Kendrick Hang is an unfunded collaborator for this project.  \textbf{Benefits to Partner:} AppConnect NW is partnering with CTCtoAI because of the direct benefit to their students. AppConnect NW's constituent CTCs have difficulty finding experiential learning activities such as internships for their students.


\subsubsection{Washington State Board for Community \& Technical Colleges (SBCTC)}

SBCTC is the Washington State board that advocates, coordinates, and directs Washington State's system of 34 public community and technical colleges. \textbf{Responsibilities:} SBCTC will connect the project with educators and students at its member CTCs; coordinate efforts between the project and the Equity in CS workgroup; and foster a STEM community of practice for CS instructors. \textbf{Benefits to Partner:} This project will help SBCTC in its goals of helping existing CTC students gain employment in CS and emerging technologies, and enabling more CTCs across the state to start successful CS degrees.


\subsubsection{Washington Technology Industry Association (WTIA)}

WTIA is a consortium serving more than 1,000 employers in Washington State that rely heavily on a Computer Science and Information Technology workforce. \textbf{Responsibilities:} WTIA and its constituent companies will provide advice on the design and implementation of the program in order to align the experiential learning component with industry needs and to align the career mentoring component with industry hiring practices. Additionally, WTIA will assist in recruiting mentors for the program. \textbf{Benefits to Partner:} WTIA is motivated to work with CTCtoAI because its constituent companies have expressed a need for a workforce trained in AI/ML in order to remain competitive.


\subsubsection{Washington State Workforce Training and Education Coordinating Board (WTB)}

The WTB is a state agency based on a partnership of business, labor, and government dedicated to helping Washington residents succeed in family-wage jobs while meeting employer needs for skilled workers. \textbf{Responsibilities:} WTB will disseminate resources and reports to other emerging technology sectors and broader audiences (such as rural areas, tribal colleges, and formerly incarcerated individuals) within Washington State. \textbf{Benefits to Partner:} The project will help WTB accomplish its goals to help employers build their workforce and to help Washington State workers become more competitive candidates for high-paying jobs and drive an inclusive economic recovery.


\subsubsection{Community College Baccalaureate Association (CCBA)}

CCBA is a national organization that hosts conferences, conducts research, and provides support and resources to community colleges that build and sustain career-focused baccalaureate degrees. \textbf{Responsibilities:} CCBA will provide a venue for the PIs to disseminate the results from the initiative each year at their annual conference. \textbf{Benefits to Partner:} CCBA aims to get more CTCs to start successful career-connected CS degrees, including in emerging technologies, and this project will provide valuable resources that CTCs nationwide can implement.

\subsection{Communication Plan}

\textit{Planning and Advisory Meetings:} AppConnect NW, as the primary implementation collaborator for this project, will meet bi-weekly with the PIs in order to coordinate logistics, troubleshoot individual student issues, and provide advice and feedback. Advisory meetings with BIL, SBCTC, and WTIA in order to share best practices, and align with workforce and CTC needs, will be conducted both ad-hoc and quarterly. The project will report back to all partners on results in the summer.

\textit{Student Recruitment from CTCs:} CTCtoAI will work with CTCs in the fall to recruit students for the program. Recruitment of 50 students annually is not expected to be a concern: student interest in such a program from AppConnect NW schools alone already exceeds the number of spots this project will make available.

\textit{Mentor Recruitment from Industry Partners:} Program staff will communicate with WTIA ad-hoc in the Fall in order to recruit mentors for career mentorship and the first micro-internship, and again in the winter to recruit mentors for the second internship. Any shortfall in the number of mentors can be made up by CodeDay and MinT's existing network of mentor volunteers with AI experience (documented in Facilities, Equipment, and Other Resources).

\textit{Dissemination Partners:} CCBA, WTB, and SBCTC will receive an annual report, as well as publications, reports, and resources as they are created, for dissemination to CTCs. The PIs will share results each year at the CCBA conference. (More details are provided in the dissemination plan provided in "Generation of Knowledge".)