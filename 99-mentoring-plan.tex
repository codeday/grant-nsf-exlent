%
% No references allowed
%
\section{Mentoring Plan}

The project will expand Mentors in Tech (MinT)'s existing mentoring program and curriculum by incorporating students served by this project. MinT offers a structured program and a tailored curriculum that closes the student support and social capital gap to larger, more established and well-known CS programs. MinT builds meaningful long-term industry mentor relationships and offers resources, industry connections, and an inclusive community of students and industry mentors. MinT is able to help the diverse student population equitably, and inclusively, gain access to experienced industry mentors and industry opportunities.

The end goal for the MinT program is for students to be able to successfully prepare a student’s transition from a student to a technology professional within 6 months of graduation, and land a role in tech.

Because the diversity of MinT students is important, the MinT curriculum for students is flexible and includes 12 monthly modules that the student and mentors can self-pace depending on where the student is in their process of entering emerging tech.  Student-facing program elements totalling 40 hours annually per student:

\begin{itemize}
    \item Each student is trained in how to be a mentee in the MinT program and then matched with two trained industry mentors based on student preference, and mentors and student background match.

    \item Each student meets with each mentor 1:1 once a month from November to June, and writes a structured reflection for each meeting afterwards.

    \item MinT provides a curriculum and resources for each meeting including mock technical interviews, LinkedIn, behavioral interviews, etc. for the whole academic year.

    \item Students attend a monthly MinT workshop and optional office hours with experienced industry recruiters.

    \item Students submit 3 job applications every month.
\end{itemize}

\subsection{Orientation and Networking}

Both mentors and mentees are trained before the start of the program  and the mentor-mentee matching process. Topics in training include:

\begin{itemize}
    \item \textit{Mentorship goals:} What the mentee would like to get out of the program and what their definition of success looks like. Ultimately, with the help of their mentors, they are able to launch their careers in emerging tech. 

    \item \textit{Expectations} of the programs are set as well as the roles of a mentor and mentee, what the mentor-mentee relationship in the context of the program entails, the successful completion of components of the program that will help the mentees reach their career goals in tech.

    \item \textit{Communications norms:} the structure and curriculum for the program and how mentees and mentors will communicate, establish regular recurring one-on-one mentorship meetings virtually. How to ask the right questions and type of questions.

    \item \textit{Conflict resolution:} How to listen, especially to each other with different backgrounds and experiences. MinT program code of conduct, and support from MinT staff.
    
    \item \textit{Confidentiality:} a range of scenarios that mentors and mentees might face when interacting in the course of the program.
\end{itemize}

The networking component of mentorship is included in their 16 one-hour one-on-one meetings that take place online, as well as in cohort activities: the online LinkedIn discussion group, monthly workshops and panels, and employer information sessions.

\subsection{Individual Development Plan and Competencies for a Successful Career}

The IDP is jointly managed by the mentor and mentee depending on where the mentee is in their career journey as well as their year in school. There are 3 tracks of curriculum topics that the mentor-mentee team can select. They can also build their own plan and monthly calendar depending on what is happening in the mentee job search process. For example, they can move up the Mock Interview meetings if they get an interview or skip ahead to offer and negotiations if they have an offer in progress.

Meeting topics include: Tech lay of the land, how tech hires, the job search process, resume review, tech interviews, managing your network, culture and conflict at work, offers and negotiation, working as an employee and managing up, planning your next career step, managing money, and final commitments and next steps after MinT program.

After each mentorship meeting, both the mentor and mentee will have an opportunity to write a reflection and fill out a structured reflection form and status which is reviewed by MinT staff as well as partner faculty.

As a part of the IDP and the 16 one-hour one-on-one mentor meetings, students are able to develop competencies for a successful career in emerging technologies including career and self-development, critical thinking (including technical and behavioral interviews), Equity and Inclusion (working as an employee and culture at work), leadership, professionalism, teamwork, and technology used in industry.

\subsection{One-on-One Meetings}

One-on-one mentor-mentee meetings are the core of the MinT program. There are 16 of them from November to June. A student has two (2) mentors that they choose that will be with them for the entirety of the school year. Having two mentors that they can meet one-on-one each month with different backgrounds and experiences exposes the student mentee to different aspects and points of view in a wide variety of roles, companies, and experiences that mentors can share even on the same meeting topic. This contributes to the overall preparation for the students to step into a very wide career field in emerging tech.
