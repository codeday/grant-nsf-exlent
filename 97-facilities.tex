%
% No references allowed
%
\section{Facilities, Equipment, and Other Resources}
\footnotetext[1]{Letter of collaboration is enclosed.}
\footnotetext[2]{Biographical sketch, Current and Pending (Other) Support, and Collaborators and Affiliations are enclosed.}

\begin{tabularx}{\textwidth}{L{0.4} L{0.7}}
    \toprule
    \textbf{Resource} & \textbf{Description} \\
    \midrule \endhead
    \midrule\multicolumn{2}{r}{\itshape continues on next page} \\
    \endfoot
    \bottomrule\endlastfoot
    
    \rowcolor{black} \color{white} Facilities & \\ \addlinespace[8pt]
    
    Staff Offices &
    The PI, co-PI, and other project staff have furnished offices/workspaces which are appropriate to their positions. \\  \addlinespace[8pt]

    \rowcolor{black} \color{white} Major Equipment & \\ \addlinespace[8pt]

    Student Computers &
    Students who participate in the program are expected to use their own computers, but it is likely that some students may not have a computer, their computer may break during the program, or their computer may not be of adequate performance to attend meetings or perform development work. CodeDay has a supply of approximately 200 laptops (Intel Core i5 with 16GB RAM, 16GB of RAM, a 256GB SSD, 13 in. display) which can be given to students, to keep. \\ \addlinespace[8pt]
    
    \rowcolor{black} \color{white} Computers & \\ \addlinespace[8pt]
    
    Canvas Learning Management System (LMS) &
    Canvas is an open-source LMS with automated task management, on-campus IT support, goal/outcome measurement, graphic analytics reporting, iOS/Android app center, and multimedia storage. CodeDay has a Canvas deployment available for students to complete any lessons needed for program or project onboarding and training. \\ \addlinespace[8pt]

    Slack &
    Slack is a cloud-based group messaging platform which will be used for cohort announcements and student/mentor peer communication. CodeDay owns a license to use Slack for any number of students. \\ \addlinespace[8pt]

    LinkedIn &
    LinkedIn is a professional social networking platform. CodeDay and MinT have established member groups for program participants to participate in discussions and connect with mentors and job opportunities. \\ \addlinespace[8pt]

    Zoom &
    Zoom is an electronic meeting platform. Both CodeDay and MinT have Zoom licenses which can be used for cohort activities and mentor meetings. \\ \addlinespace[8pt]

    CodeDay Program Management Software &
    CodeDay uses a custom, open-source software suite to manage student participation in micro-internships. The software is capable of: receiving student applications to participate in the program; collecting application reviews and conducting admissions; collecting student preferences and matching students to projects; conducting surveys and collecting upward, downward, and peer feedback; collecting daily ``stand-up'' updates from students; and scoring and flagging at-risk students automatically based on AI. \\ \addlinespace[8pt]
    
    MinT Program Management Software &
    MinT uses an in-house software suite to manage student career mentorship and cohort activities. The software is capable of: collecting student preferences and matching students to mentors; tracking mentor meetings; collecting structured reflections from students and mentors; advertising cohort activities; and flagging of at-risk students for staff follow-up. \\ \addlinespace[8pt]

    \rowcolor{black} \color{white} Institutional Collaborators & \\
    \multicolumn{2}{l}{\textit{Note: Only those partners that will routinely contribute to project activities are listed.}} \\ \addlinespace[8pt]
    
    AppConnect NW (DUE\#1700629)\footnotemark[1] &
    AppConnect NW is a consortium of nine Community and Technical Colleges in Washington State, created to bridge the gap between community colleges and the technology industry. \newline\vspace{8pt}
    AppConnect NW has committed to connecting the project with educators and students at its member CTCs, and will work closely with the PIs in an advisory role.  \\ \addlinespace[8pt]

    Washington State Board for Community \& Technical Colleges (SBCTC)\footnotemark[1] &
    The SBCTC is the Washington State board which advocates, coordinates and directs Washington State's system of 34 public community and technical colleges. \newline\vspace{8pt}
    SBCTC has committed to: connecting the project with educators and students at its member CTCs; coordinating efforts between the project and the Equity in Computer Science workgroup; and fostering a STEM community of practice for CS instructors. \\ \addlinespace[8pt]

    Community College Baccalaureate Association (CCBA)\footnotemark[1] &
    CCBA is a national organization that hosts conferences, conducts research, and provides support and resources to community colleges that build and sustain career-focused baccalaureate degrees. \newline\vspace{8pt}
    CCBA has committed to supporting dissemination efforts within CTCs by having MinT and CodeDay present project findings to researchers and college administrators, faculty, and leaders. \\ \addlinespace[8pt]

    Washington Technology Industry Association (WTIA)\footnotemark[1] &
    WTIA is a consortium serving more than 1,000 employers in Washington State that rely heavily on a Computer Science and Information Technology workforce. \newline\vspace{8pt}
    WTIA has committed to providing advice in the design and implementation of the project, and to connect the project with companies and industry experts focused on AI. \\ \addlinespace[8pt]

    Washington State Workforce Training and Education Coordinating Board (WTB)\footnotemark[1] &
    The WTB is a state agency based on a partnership of business, labor, and government dedicated to helping Washington residents succeed in family-wage jobs, while meeting employer needs for skilled workers. \newline\vspace{8pt}
    The WTB has committed to sharing the model with other emerging technology sectors in Washington State and to broader audiences (including rural areas, tribal colleges, and formerly incarcerated individuals). \\ \addlinespace[8pt]

    City College of San Francisco (CCSF)\newline\vspace{4pt}
    University of California, San Francisco (UCSF)
    \newline\vspace{4pt}Building Inclusive Labs Collaborative Project (DUE\#2055735/2055309)\footnotemark[1] &
    Building Inclusive Labs (BIL) is a collaborative initiative between a community college biotechnology program (CCSF) and a top-tier research university and technician employer (UCSF) whose goal is to build more inclusive workplace environments for community college students pursuing biotechnology careers. \newline\vspace{8pt}
    BIL has committed to: providing opportunities for community college students to participate in their Workplace Navigation Training; providing opportunities for industry partners to participate in their Inclusive Mentor-Manager Training; and fostering an experiential learning community of practice to share practices across emerging technology sectors. \\ \addlinespace[8pt]
    
    \rowcolor{black} \color{white} Employees (Personnel) & \\ \addlinespace[8pt]

    Open Source Partnership Staff &
    CodeDay employs a team to build partnerships with open source maintainers and develop a portfolio of issues for its open source summer internships. Issues which are too simple for the summer internship program will be matched with mentors and students during the micro-internship.
    \\ \addlinespace[8pt]

    TA Staff &
    CodeDay employs a team of TAs who provide technical advice and debugging help to students in its community; this team can provide help to students participating in a micro-internship.
    \\ \addlinespace[8pt]
    
    Administrative staff &
    CodeDay and MinT employ administrative, bookkeeping, and legal staff with experience managing grants. \\ \addlinespace[8pt]
    
    \rowcolor{black} \color{white} Unfunded Collaborators & \\
    \multicolumn{2}{l}{\textit{Note: Only collaborators that will significantly contribute to project activities are listed.}} \\ \addlinespace[8pt]

    Kendrick Hang \newline\vspace{4pt} \small{Instructor at Green River College and Faculty Associate in AppConnect NW (DUE\#1700629)}\footnotemark[1,2] &
    Kendrick Hang is a full-time teaching faculty member at Green River College, teaching Information Technology and Computer Science, and also serves as a Faculty Associate in The Northwest Network for Application Development and Technology Connections (AppConnect NW), a consortium of nine community and technical colleges in Washington State offering baccalaureate degree programs in software development and computer science. \newline\vspace{8pt}
    Kendrick will serve as a liaison to community and technical college faculty members and administrators, assisting in project efforts: recruiting students, identifying curriculum alignment opportunities, collaborating on program design and improvement, advising on evidence-based inclusive practices, supporting data collection/evaluation needs, and contributing to project dissemination activities. \\ \addlinespace[8pt]
    
    \rowcolor{black} \color{white} Other & \\ \addlinespace[8pt]

    Mentors &
    CodeDay and MinT have an existing pool of 178 active mentors who have experience in AI/ML who could serve as mentors for the program in addition to those recruited through WTIA. \\ \addlinespace[8pt]

    Micro-Internship Onboarding Curriculum &
    CodeDay has developed an onboarding curriculum to prepare introductory CS students for micro-internships which may be used or adapted for the CTC AI micro-internships proposed by this project. \\ \addlinespace[8pt]

    CS Career Mentoring Outlines &
    MinT has developed meeting and topic outlines for career mentor meetings which may be used or adapted for the mentoring proposed by this project.
    
\end{tabularx}

\footnotetext[1]{Letter of collaboration is enclosed.}
\footnotetext[2]{Biographical sketch, Current and Pending (Other) Support, and Collaborators and Affiliations are enclosed.}