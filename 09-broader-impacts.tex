\section{Broader Impacts}

Although most students pursue post-secondary education to improve their job prospects, \cite{alshahraniUsingSocialCognitive2018,helpsStudentExpectationsComputing2005,nortonPerceivedBenefitsUndergraduate2017} many new graduates report they are unprepared for a career,  especially those in the fast-evolving emerging tech fields.\cite{begelStrugglesNewCollege2008,craigListeningEarlyCareer2018,kapoorUnderstandingCSUndergraduate2019} Likewise, employers in the emerging field of AI/ML have reported that new graduates lack the skills to succeed.\cite{begelStrugglesNewCollege2008} Industry engagement programs such as mentorship and internships have been shown to close these skill gaps, as well as increase familiarity with the field and improve long-term retention. \cite{beaubouefComputerScienceCurriculum2011,frylingCatchEmEarly2018,tashakkoriEarlyParticipationCS2011,dahlbergImprovingRetentionGraduate2008,salinitriEffectsFormalMentoring2005,dennehyFemalePeerMentors2017,campbellFacultyStudentMentor1997b,mangoldWhoGoesWho2002} However, traditional models for these programs are expensive to scale due to the large upfront commitments required for companies as well as colleges. As a result, only elite universities\cite{cohenUsingInternalInternship1995} and very large tech companies offer these experiences to few students.

These problems are larger for students enrolled in CTCs, which have fewer resources, industry relationships, and alumni base. This proposal gives access to an incredibly diverse CTC student population wrap-around mentoring and internship support, which previously only been widely available to their traditional 4-year counterparts and builds a pathway to successful AI/ML career pathways previously not available to most CTC students. 

Over three years, the project will provide mentorship and internships to 150 students as well as highly competitive wages (which are likely higher than what many students earn in their day jobs). A sustainable program at scale and published details enabling others to replicate the program will provide these experiences to thousands of students in the long term.

\subsection{Developing a Globally Competitive Workforce}

Internships provide critical hands-on experience that helps students understand how to apply what they have learned and prepare to join the workforce. They are also the largest factor in determining whether students find a job post-graduation.\cite{callananAssessingRoleInternships2004,jonesTransformingCurriculumPreparing2002,knouseRelationCollegeInternships1999,saltikoffPositiveImplicationsInternships2017,stepanovaHiringCSGraduates2021} This project's micro-internships will prepare students for AI work and build their resume to obtain jobs.

Likewise, career mentoring helps students stay engaged and take the next steps to get a job. MinT's mentoring program has been described as a ``critical ingredient to high-quality programs that produce equitable education, employment, and life outcomes for students and graduates.''\cite{bragg20PromisingPractices} Students who receive the mentoring provided in this project will be left with a clear path into AI jobs.

\subsection{Driving Full Participation of Underrepresented Populations in AI}

This program specifically targets the exceptionally diverse and under-represented group of students who attend CTCs. As discussed in the ``Building an Inclusive and Diverse STEM Workforce'' section, these students are otherwise very unlikely to obtain internships and have difficulty obtaining any technology jobs. This project will provide these under-represented students a path to upward mobility through highly in-demand AI careers.